\chapter{Создание и конвертация сеток}

Эта глава описывает все вопросы, связанные с созданием сеток в пакете OpenFOAM:
Раздел 5.1 даёт обзор путей, которыми может быть описана сетка в OpenFOAM;
 Раздел 5.3 охватывает подпрограмму blockMesh для генерации простых сеток из
блоков
гексаэдрических элементов; Раздел 5.4 охватывает подпрограмму snappyHexMesh для
автоматического создания сложных сеток из гексаэдрических и расщеплённых
гексаэдрических элементов из
 триангулированных геометрических поверхностей; раздел
5.5 описывает имеющиеся опции для конвертации сетки, созданной в других пакетах,
в формат, доступный для чтения OpenFOAM.

\section{Описание сетки}
\label{sec:5.1}

Этот раздел раскрывает описание классов OpenFOAM С++, работающих с сеткой.
Сетка является составной частью численного расчёта и должна удовлетворять
определённому
 критерию для получения качественного и точного результата. В процессе
запуска, OpenFOAM проверяет соответствие сетки набору строгих ограничений и
завершит
 работу, если ограничения не будут соблюдены. Это важно, потому что пользователь
 перед запуском OpenFOAM может некорректно исправить большую сетку
в программах сторонних разработчиков. Это является некоторым недостатком, но
мы (разработчики) не оправдываем OpenFOAM просто усваивая хорошие примеры
из практики. Убедитесть в том, что сетка соответствует требованиям расчёта; в
ином
случае решение будет некорректным уже перед запуском программы.
По умолчанию OpenFOAM определяет 3D сетку из случайных многогранников,
ограниченных случайными полигональными гранями, т.е. элементы могут иметь
бесконечное количество поверхностей, в которых, для каждой поверхности, имеется
 неограниченное количество рёбер и нет ограничений на их положение. Сетка с
такой
 общей структурой известна в OpenFOAM как polyMesh. Детально она
описывается в разделе 2.1 Руководства программиста, но важно запомнить, что
этот тип сетки позволяет получить большую свободу при создании и манипуляции
с сетками, в частности когда геометрия домена сложна, либо изменяется во времени.
 Ценой такой универсальности сетки является сложность её генерации с помощью
 программ-преобразователей. Поэтому в OpenFOAM предусмотрена утилита
cellShape для управления общепризнанными форматами сеток, основанных на множестве
 предопределённых форм элементов.

\subsection{Описание сетки и её применимость}

Перед описанием OpenFOAM-формата сетки, polyMesh, и утилит cellShape, мы, в
первую очередь, установим критерии применимости, используемые в OpenFOAM.
Условия, которым должна удовлетворять сетка даны ниже:

\subsubsection{Точки}

Точка является областью 3D пространства, определяемой вектором в метрах. Точки
компилируются в список и каждая точка связывается с указателем, который обозначает её
 позицию в списке, начинающемся с нуля. Список точек не может содержать
две точки, обладающие совершенно идентичным расположением, так же как точку,
не принадлежащую по крайней мере одной грани.

\subsubsection{Грани}

Грань - это упорядоченный список точек, в котором каждая точка имеет свою метку.
Упорядочивание меток точек происходит из условия того, чтобы две соседние точки
 были соединены ребром, т.е. следуя нумерации вы перебираете точки, обходя по
кругу грани. Грани компилируются в список и каждая грань имеет свою метку, отображающую
 её положение в списке. Направление вектора нормали грани выбирается
по правилу правой руки, т.е. при взгляде по направлению к грани, если нумерация
точек идёт против часовой стрелки, нормаль направлена к Вам, как показано на
рисунке 5.1.

