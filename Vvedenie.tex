\addcontentsline{toc}{chapter}{Права}

Copyright \textcopyright2011 Фонд OpenFOAM.

Разрешение предоставлено копировать, распространять и/или изменять этот документ согласно пунктам из GNU
 Свободной Лицензии Документации, Версия 1.2, изданная Фондом Свободного Программного обеспечения; без Неизменяемых Секций,
 никаких обложек Текстов Обратной стороны и ни одной Лицевой Обложки Текста (охраняемых этой лицензией):
“Доступно свободно на сайте ”openfoam.org”. Копия лицензии включена в секцию названную “GNU Свободная Лицензия Документации”.

Этот документ распространен в надежде, что будет полезен, но БЕЗ КАКОЙ-ЛИБО ГАРАНТИИ; без даже подразумеваемой гарантии
 ВЫСОКОГО СПРОСА или ПРИГОДНОСТИ В ПРИКЛАДНЫХ ЦЕЛЯХ.

Набран в LATEX.





(Свободная лицензия для Документации)

Версия 1.2, ноябрь 2002 г.
Copyright ©2000,2001,2002 Free Software Foundation, Inc.

59 Temple Place, Suite 330, Бостон, MA 02111-1307 США

Каждому разрешается копировать и распространять дословные копии этого документа лицензии, но изменение ее (Лицензии)
 не допускается.

Предисловие
Назначение этой Лицензии состоит в том, чтобы сделать руководство, учебник, или иной деловой и полезный документ
 "свободным" под свободой подразумевается: гарантировать каждому эффективную свободу копировать и распространять его,
 с или без изменений его, либо коммерчески либо некоммерчески. Во-вторых, эта Лицензия сохраняет для автора и издателя
 способ получить уважение за их работу, не будучи преследуемыми и ответственными за изменения, сделанный другими.
Эта Лицензия - своего рода “copyleft”, что означает что производные работы над документом должны сами быть свободными
 в том же самом смысле. Это служит дополнением Общей Публичной Лицензии GNU, которая является лицензией copyleft,
 разработанной для бесплатного программного обеспечения.
Мы разработали эту Лицензию, чтобы использовать ее для руководств для бесплатного программного обеспечения, потомучто
 бесплатное программное обеспечение нуждается в бесплатной документации: бесплатная программа должна идти с руководствами,
 имеющими те же самые свободы, которые имеет программное обеспечение. Но эта Лицензия не ограничена применением лишь для
 руководств для программного обеспечения; она может быть использована для любой текстовой работы, независимо от того
 публикуется ли она в качестве печатного издания. Мы рекомендуем эту Лицензию преимущественно для работ, цель которых -
 инструкция или литературная ссылка.

1. Применимость и определения
Эта Лицензия применима к любому руководству или иной работе, выполненной в любой среде, которая содержит упоминание
 о Лицензии, помещенное держателем авторского права, как указание, что работа может распространятся в соответствии с
 данной Лицензией. Такое уведомление предоставляет международную, лицензию, без лицензионного платежа, неограниченную
 по времени, чтобы использовать документ в работе при условиях, установленных здесь. Термин "Документ", используемый
 здесь и здесь, обозначает любое  руководство или работу.
Любой автор - лицо, имеющее патент, будем обозначать "Вы". Вы принимаете лицензию, если Вы копируете, изменяете или
 распространяете работу которая требует получения разрешения согласно закону об авторском праве (copyright).
“Модифицированная Версия” Документа есть любая работа, содержащая Документ или часть его, либо дословную копию, либо
 с изменениями и/или переводом на другой язык.
“Вторичная Секция” есть озаглавленное приложение или секция вступительной части Документа, которая описывает отношение
 издателей или авторов Документа к полному предмету Документа (или связанным делам) и не содержит ничего, что могло бы
 попасть непосредственно в предметную область описываемого объекта. (Таким образом, если Документ, например, учебник по
 математике, то Вторичная Секция не может не содержать никакой математики). Во вторичной секции может описываться
 историческая связь с предметом или с вопросами, связанными с предметом, или касающихся легального юридического,
 коммерческого, философского, этического или политического положения.
“Инвариантные Секции” это Вторичные Секции, содержание которых говорит, что Документ выпущен согласно этой Лицензии.
 Если секция не делает разъяснение вышеупомянутому определению Вторичных Секций, тогда она не может определяться как
 Инвариант. Документ может содержать пустые Инвариантные Секции. Если Документ совсем не определяет каких либо Инвариантных
 Секций тогда в нем их нет.
“Тексты Обложек (Покрытия)” это определенные короткие элементы текста, которые перечислены, как Текст Лицевой Обложки
 или Текст Обратной Обложки, причем наличие их указывается там же, где и то, что Документ выпущен согласно этой Лицензии.
 Текст Лицевой Обложки может содержать не более 5 слов, а Текст Обратной Обложки может содержать не более 25 слов.
"Прозрачная" копия Документа означает машино - читаемую копию, представленную в формате, спецификация которого общедоступна,
 является подходящей для просмотра документа непосредственно с помощью использованных при создании документа  редакторами
 текста или (в случае изображений, составленных из пикселов) аналогичными програмами рисования или некоторыми широко
 доступными редакторами рисунков, что является подходящим для текстовых процессоров или для автоматического преобразования
 в различные форматы, удобные для ввода в текстовые процессоры. Копия сделанная в противоположном Прозрачному формате,
 использование которого, или невозможность использования, с целью помешать, или воспрепятствовать последующей модификации
 читателями не является Прозрачной копией. Формат изображения не Прозрачен если используется любое существенное количество
 текста. Копию, которая не "Прозрачна", называют "Непрозрачной".
Примеры подходящих форматов для Прозрачных копий включают простой ASCII текст без форматирования, формат ввода Texinfo,
 LaTex, SGML или XML использующие для всех доступные DTD, и удовлетворяющие стандарту простого HTML, PostScript или PDF,
 в версиях, разработанных для модификации.
Примеры прозрачных форматов изображения включают PNG, XCF и JPG. Непрозрачные форматы включают использующие право
 собственности, которые могут быть прочитаны и отредактированы только при наличии лицензии на соответствующие текстовые
 процессоры, SGML или XML, для которого DTD и/или инструменты обработки не общедоступны, и  автоматизированно
 сгенерированный HTML, PostScript или PDF, произведенный с помощью текстовых редакторов (текстовыми процессорами)
 только для чтения (без возможности редактирования человеком).
"Титульный лист" обозначает, для печатной книги, непосредственно  титульный лист, а также следующие за ним страницы,
 если они необходимы, чтобы содержать, легально и четко, материал, которого  требует эта Лицензия, и который должен 
находиться на титульном листе. Для работ в форматах, которые не имеют никакого титульного листа,  "Титульный лист" 
также текст около самого видного появления названия работы, непосредственно до начала текста
Секция “Озаглавленный XYZ” обозначает название секции Документа, название которое или - просто XYZ или содержит XYZ 
в круглых скобках после текста, который переводит XYZ на другой язык. (Здесь XYZ обозначает особое название секции,
 упомянутое ниже, типа "Благодарностей", "Посвящений", "Одобрений", или "Истории"). Чтобы “Сохранить Название” такой
 секции, когда Вы изменяете назначение Документа, то она остается секцией “Озаглавленный XYZ” согласно этому определению.
Документ может включать Гарантийные Правовые обязательства вместе с уведомлением о том, что эта Лицензия применена к
 Документу. Эти Гарантийные Правовые обязательства, рассматриваются как ссылающиеся на эту Лицензию, но только, что
 касается отказа от гарантий: любое другое значение, которое эти Гарантийные Правовые обязательства могут иметь,
 недействительно и не имеет никакого эффекта на значение самой Лицензии.

2. Дословное копирование 
Вы можете копировать и распространять Документ в любой среде,  коммерчески или некоммерчески, при условии включения
 этой Лицензии, уведомления об авторском праве и уведомления о лицензии, включая примечания к Лицензии при обращении
 к Документу, воспроизведенных во всех копиях, и Вам не разрешается добавлять никаких других условий кроме перечисленных
 в этой Лицензии. Вам не разрешается использовать технические методы, чтобы затруднить или управлять чтением или далее
 созданием копий, которые Вы делаете или распространяете. Однако, Вы можете просить компенсацию в обмен на копии. Если
 Вы распространяете достаточно большое число копий, Вы должны также следовать условиям, изложенным в секции 3.
Вы можете также предоставить копии, при тех же самых вышеизложенных условиях и можете демонстрировать копии.

3. Количество копий
Если Вы издаете печатные копии (или копии в СМИ, которые обычно имеют печатные обложки) Документа, числом больше 100 штук,
 и уведомление лицензии Документа требует Текстов Обложек, Вы должны включить в копии Обложки, которые передадут, ясно и
 четко, все эти Тексты Обложек: Текст Лицевой Обложки на передней обложке, и Текст Задней Обложки на обратной стороне.
 Обе обложки должны также ясно и четко идентифицировать Вас как издателя этих копий. Лицевая Обложка должна содержать
 полное название - все слова названия, видных одинаково. Вы можете добавить другой материал на обложки, кроме существующего.
 Копирование с изменениями, ограничивающихся обложками, пока они сохраняют название Документа и удовлетворяют всем условиям,
 можно рассматривать как дословное копирование во всех отношениях.
Если необходимые тексты для любой обложки слишком велики, чтобы поместиться полностью, Вы должны поместить сначала 
перечисленные (столько, сколько разумно поместится) на фактической обложке, и продолжить размещение остальных на следующих
 страницах.
Если Вы издаете или распространяете Непрозрачные копии Документа, количеством более чем 100 штук, Вы должны либо включить
 машиночитаемую Прозрачную копию наряду с каждой Непрозрачной копией, либо вставить в или с каждой Непрозрачной копией,
 ссылку на местоположение в компьютерной сети, к которой каждый использующий сеть интернет человек имеет доступ, чтобы
 скачать, используя протоколы общедоступной стандартной сети,  полную Прозрачную копию Документа, без постороннего материала.
 Если Вы используете второй вариант, Вы должны принять меры, когда начинаете распространение Непрозрачных копий в количестве,
 гарантирующем, что эта Прозрачная копия останется таким образом доступной в установленном местоположении по крайней мере
 один год с момента, когда Вы в последний раз распространяете Непрозрачную копию (непосредственно или через ваших агентов
 или розничных продавцов) того выпуска.
Это требует, но не требуется, что бы Вы связывались с авторами Документа задолго до распределения любого большого количества
 копий, чтобы давать им шанс предоставить Вам обновленную версию Документа.

4. Изменения
Вы можете копировать и распространять измененную Версию Документа при условиях описанных выше в секциях 2 и 3, при условии,
 что Вы выпускаете измененную Версию под этой же Лицензией, с измененной Версией выполняющей роль Документа, таким образом
 лицензируя распределение и модификацию измененной Версии для кого бы то ни было, кто бы не получил копию. Кроме того,
 Вы должны выполнить следующие пункты в модифицированной Версии:
A. Использовать в Титульном листе (и на обложках, если имеются) название, отличное от предудыщего Документа, и от предыдущих
 версий (которые, если были, должны быть перечислены в разделе «истории» Документа). Вы можете использовать то же самое
 название как в предыдущей версии, если оригинальный издатель прежней версии дает разрешение.
B. Список авторов на Титульном листе, одного или более или субъектов, ответственных за изменения в модифицированной Версии,
 вместе с по крайней мере пятью основными авторами Документа (все его основные авторы, если их менее чем пяти), если они не
 освобождают Вас от этого требования.
C. Указать на Титульном листе название издателя модифицированной Версии, как издателя (издательство).
D. Сохранить все уведомления об авторском праве Документа.
E. Добавить соответствующее уведомление об авторском праве для ваших модификаций смежного с другими уведомлениями об
 авторском праве.
F. Включить, немедленно после уведомлений об авторском праве, уведомление о лицензии, дающее публичное разрешение
 использовать измененную Версию в соответствии с этой Лицензией, в форме, показанной в Приложении ниже.
G. Сохранить в этой лицензии замечание о полном списке неизменяемых Инвариантных Секций и требуемых Текстов Обложек,
 перечисленных в уведомлении о лицензии Документа.
H. Включить неизмененную копию этой Лицензии.
I. Сохранить секцию, названную "История", чтобы сохранить ее Название, и добавить к этому пункту по крайней мере имеющееся
 название, год выпуска, новых авторов, и издателя Модифицированной  Версии, как указано в Титульном листе. Если нет
 никакой секции, названной "История" в Документе, создайте одну, устанавливающую название, год, авторов, и издателя Документа
 как было на его Титульном листе, затем добавьте пункт, описывающий Модифицированную Версию как установлено в предыдущем
 предложении.
J. Сохраните ссылку о местоположении в сети интернет, если таковая вообще имеется, данные для открытого доступа к Прозрачной
 копии Документа, и аналогично местоположения сети, данные в Документе для предыдущих версий на которых они были основаны.
 Они могут быть размещены в секцию "Истории". Вы можете опустить местоположение сети для работы, которая была издана по
 крайней мере за четыре года до Документа непосредственно, или если оригинальный издатель версии, на которого оно ссылается,
 дает разрешение.
K. Для любой секции, названной "Благодарности" или "Посвящения", сохраните Название секции, и сохрание в секции все
 содержание и тон каждого написавшего благодарности и/или посвящения.
L. Сохраните все Инвариантные Секции Документа, неизменными как по тексту так и по названию. Номер секции или эквивалент
 не считаются частью названий секции.
M. Удалите любую секцию названную "Одобрения". Такая секция не может быть включена в Модифицированную Версию.
N. Не делайте изменение названия для любой существующей секция, которая называется "Одобрения" или имен конфликтующих
 в названии с любой Инвариантной Секцией.
O. Сохраняйте все Гарантийные Правовые обязательства (отказ от обязательств).

Если Модифицированная Версия включает новые секции во вступительной части или приложениях, которые готовятся как Вторичные
 Секции и не содержат никакого материала, скопированного с Документа, Вы можете, по вашему выбору определить некоторых или
 все эти секции как инвариант. Чтобы сделать это, добавьте их названия в список Инвариантных Секций в уведомлении лицензии
 Модифицированной Версии. Эти названия должны быть отличными от любых других названий секций.
Вы можете добавить секцию названную "Одобрения", если она не содержит ничего кроме одобрения вашей Модифицированной Версии
 различными сторонами, например, утверждения аналогичного обзора  или того, что текст был одобрен организацией - как
 авторитетное определение стандарта.
Вы можете добавить часть из не более пяти слов как Текст Лицевой Обложки, и текст не более 25 слов как Текст Задней Обложки,
 в конец списка Текстов Обложек в Модифицированной Версии.
Только один фрагмент Текста Лицевой Обложки и один Текст Задней Обложки может быть добавлен либо через операции,
 выполненные любым юридическим лицом. Если Документ уже включает текст обложки для обложки, предварительно добавленной
 Вами, или в соответствии с встречей, назначенной тем же самым юридическим лицом от имени которого Вы действуете, Вы не
 можете добавлять еще одну; то Вы можете заменить старую, получив явное разрешение от предыдущего издателя, который добавил
 старую обложку.
Автор(ы) и издатель(и) Документа, в соответствии с этой Лицензией, не дают разрешение использовать их имена для опубликования
 или для утверждения или требования одобрения для любой Модифицированной Версии.

5. Объединение документов
Вы можете объединить Документ с другими документами, выпущенными согласно этой Лицензии, согласно терминам определенным
  в секции 4 выше для модифицированных версий, при условии, что Вы включаете в комбинацию все Инвариантные Секции всех
 оригиналов документа, неизмененными, и перечисляете их всех как Инвариантные Секции вашей объединенной работе в ее
 уведомлении лицензии, и что Вы сохраняете все их Гарантийные Правовые оговорки.
Объединенная работа должна только содержать одну копию этой Лицензии, и многократные идентичные Инвариантные Секции могут
 быть заменены единственной копией. Если есть многократные Инвариантные Секции с тем же самым названием, но отличающимся
 содержанием, сделайте уникальным название каждой такой секции, добавляя в конце ее, в круглых скобках, название
 оригинального автора или издателя той секции если оно известно, или иначе уникального номера версии. Сделайте то же
 самое регулирование названий секции в списке Инвариантные Секции в уведомлении лицензии о объединенной работе. 
В комбинации документов, Вы должны объединить любые секции Называющиеся "История" в различных оригиналах документа,
 формируя одну секцию Называющуюся "История"; аналогично объедините любые секции Называющиеся "Благодарности", и любые
 секции Называющиеся "Посвящения". Вы должны удалить все секции Названные "Одобрения". (Прим. Перев. - являющиеся прямо
 или косвенно саморекламой предыдущих авторов).

6. Коллекция документов
Вы можете сделать собрание, коллекцию, состоящую из Документа и других документов выпущенных согласно этой Лицензии, и
 заменить индивидуальные копии этой Лицензии в различных документах с единственной копией, которая включена в собрание,
 при условии, что Вы следуете за правилами этой Лицензии для дословного копирования каждого из документов во всех других
 отношениях.
Вы можете извлечь единственный документ из такого собрания, и передавать его индивидуально согласно этой Лицензии, если
 Вы вставляете копию этой Лицензии в извлеченный документ, и следуете за этой Лицензией во всех других отношениях
 относительно дословного копирования того документа.

7. Совокупность независимых работ
Компиляцию Документа или его производных с другими отдельными и независимыми документами или работами, в или на объеме
 среды хранения или распределения, называют "совокупностью", если авторское право, следующее из компиляции не используется,
 чтобы ограничить юридические права пользователей компиляции вне того, чтобы разрешить человеку работать. Когда Документ
 включен в совокупность, эта Лицензия не обращается к другим работам в совокупности, которые не являются непосредственно
 производными работы Документа.
Если требование Текста Обложки секции 3 применимо к этим копиям Документа, то, если Документ - меньше чем одна половина
 всей совокупности, Тексты Обложки Документа могут быть помещены в обложки, которые заключают в скобки Документ в пределах
 совокупности, или электронного эквивалента обложек, если Документ находится в электронной форме. Иначе они должны
 появиться на печатных обложках, которые заключают в скобки целую совокупность документов.

8. Перевод 
Перевод считают своего рода модификацией, таким образом Вы можете передавать переводы данного документа в соответствии
 с секцией 4. Замена Инвариантных Секций с переводами требует специального разрешения от их держателей авторского права,
 но Вы можете включить переводы некоторых или всех Инвариантных Секций в дополнение к оригинальным версиям этих
 Инвариантных Секций. Вы может включить перевод этой Лицензии, и всех уведомлений лицензии в Документе, и любых
 Гарантийных Правовых обязательствах, при условии, что Вы также включаете оригинальную английскую версию этой Лицензии
 и оригинальные версии тех уведомлений и правовых оговорок. В случае разногласия между переводом и оригинальной версией
 этой Лицензии или уведомления или правовых обязательств от оригинала, версия оригинала будет преобладать.
Если секция в Документе называется "Благодарности", "Посвящения", или "История", требование (секция 4), чтобы Сохранить
 ее Название (секция 1) будет типично требовать изменения фактического названия.

9. Прекращение лицензии
Вы не можете копировать, изменять, сублицензировать, или распространять Документ кроме способов явно обеспеченных
 согласно этой Лицензии. Любая другая попытка скопировать, чтобы изменить, чтобы сублицензировать или распределять
 Документ является (юридически пустой) недействительной, и автоматически закончит ваши права согласно этой Лицензии.
 Однако, стороны, которые получили копии, или права, от Вас согласно этой Лицензии не будут заканчивать их лицензии,
 пока такие стороны остаются в полном согласии.

10. Будущие ревизии этой лицензии (GNU)
Свободный Фонд Программного обеспечения может издать новые, пересмотренные версии лицензии GNU Свободной Лицензии
 Документации время от времени. Такие новые версии будут подобны по духу существующей версии, но могут отличаться
 в деталях, чтобы удовлетворять новым проблемам или беспокоящим моментам. См. http://www.gnu.org/copyleft/.
Каждой версии Лицензии дают различающий номер версии. Если Документ определяет, что специфическая пронумерованная
 версия этой Лицензии “или любой более поздней версии” обращается к нему, Вы имеете выбор следовать за терминами и
 условиями или этой данной версии или любой более поздней версии, которая была издана (не как предварительный проект)
 Свободным Фондом Программного обеспечения. Если Документ не определяет номер версии этой Лицензии, Вы можете выбрать 
любую версию когда-либо издаваемую (не как предварительный проект) Свободным Фондом Программного обеспечения.